\chapter{Oldrunner -- Atari OS Rev. B}
The Atari 400/800 is the only 8-bit home computer class that contains a
well-designed and structured operating system (OS). This allows for changes,
extensions and improvements without the need to adapt current programs.
That's why Atari could use a more sophisticated OS in the later Atari XL/XE
series. Annoyingly, there are programs (from 1980-1983) that don't run on the XL
series. This is the fault of the programmers who did not stick to the official
programming guidelines.

To be able to use these incompatible programs on the Atari XL/XE, so called
\fq{Translator} disks were published. They disable the OS ROM and copy the old
OS version \fq{Rev. B} into the RAM which is located under the OS ROM.
Although this works well in most cases, it's time consuming and unpleasant to
always have to load this alternative OS. Only a hardware solution where the old
OS is stored in a kind of ROM that is always available and unmodifiable will
work with all  incompatible programs. Unfortunately these so called
\fq{Oldrunners} normally require modifications within the Atari.

With the \frz it is now possible to implement an Oldrunner without modifications
within the Atari. The memory management logic makes this possible. The Oldrunner
can be activated and deactivated using the switch labeled \fsw{OldOS}. Move the
switch to the right position \fval{ON} to activate the Oldrunner. Move the
switch only while the Atari is off. Otherwise the content of the RAM will not
match the required content of the newly selected OS version and the system will
crash.

For various reasons, it's best to only activate the Oldrunner if it is the only
way to get a program to work. While the Oldrunner is active, there is no
built-in BASIC, no RAM extension and no warm start. Pressing the \fkey{RESET} key
while the Oldrunner is active always (!) triggers a cold start and the content
of the RAM will be lost. The key which corresponds to the \fkey{RESET} key of
the Atari XL/XE was called \fkey{SYSTEM RESET} in the Atari 400/800 and
triggered a non-maskable interrupt (NMI).  It's not very difficult to retrofit
this key (see section \ref{sec:systemreset}) but it involves opening the Atari
and is therefore rather an expert task. Besides that, this key is not really
needed anyway because most of the incompatible programs are games which have
mapped the \fkey{SYSTEM RESET} to a cold start or a system crash.
This is the ridiculous attempt to annoy the \fq{crackers}, but misses the point
and only annoys the user instead.

