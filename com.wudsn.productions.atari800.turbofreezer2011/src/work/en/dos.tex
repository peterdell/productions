\section{Disk Operating System (DOS)}
Who didn't encounter the following situation: you have spent the past 3 your
editing your program and now it's time to save it again. But instead of the
expected success message you get a \fmsg{File locked} or \fmsg{Disk full} error
message. Now it's hard to know what to do. Starting the DUP via \fcmd{DOS} will
make you lose your program because you didn't activate \fq{MEM.SAV}, as it takes
forever to execute.

As the freezer is capable of stopping any program at any point, it suggests
itself to also include a DOS with the most commonly used commands.
Then it's no problem to handle the situation and continue the frozen
program afterwards.
The built-in DOS of the \frz contains all required functions to deal with the
situation mentioned above. It supports single-, enhanced-, and double-density
and is fully compatible to DOS 2.0 and DOS 2.5. The functions of the Turbo 1050
floppy speeder are fully supported, as well as the high SIO speed of
Happy/Speedy compatible disk drives and XF-551 drives. Also any other tuned or
standard disk drive can be used as well.

A disk command consists of  a three character long command or a command followed
by one or more spaces and a filename. Some commands also allow for specifying an
option that may be added to the command, separated by a slash. The command must
start immediately after the prompt. And besides the before mentioned space, no
additional spaces must be entered. File names may contain the wildcard \fq{\fmsg{*}}
replacing any character sequence, as well as \fq{\fmsg{?}} replacing any single
character.
Illegal commands will simply be ignored and will not cause an error message.
Error messages are only be displayed if they were caused by the execution of
a command.

\subsection{Executing commands}
If the prefix \fmsg{D:} is not included in a file name, the default 
\fmsg{D1:} will be used. RAM disks are not supported, as they always
depend on the DOS and RAM disk driver that is used. Because of technical
reasons PBI devices can unfortunately also not be used.
Wildcards may also be used in the destination file names. In this case, the
first matching file name is used. The only exception is the rename command
\fkey{REN} which will be described in the next section.

The commands \fcmd{DEL}, \fcmd{LOC}, \fcmd{UNL}, and \fcmd{REN} for manipulating
directory entries may operate on multiple files in a row. To prevent unwanted
actions, an option has to be added to have the command be effective on multiple
files with similar names. If no option is added, only the first matching file
will be processed. The option \fcmd{/Q} will ask for confirmation with \fkey{Y}
for every matching file name. Pressing any other key than \fkey{Y} will skip
the file. Stopping the command is possible using the \fkey{BREAK} key.
And if you're sure, use the option \fcmd{/A} to process all files with matching
file names without confirmation.
Errors that occur during execution will be displayed in readable text and will
result in the return to the freezer main menu.

\subsection{Command summary}
The commands are listed here in table form only, as they are not new or
unknown. Beginners may look up the commands in any DOS manual, like the one that
comes with the Atari 1050 disk drive.

\begin{fcmdlist}
\item[DIR]
List directory with all files
\item[DIR filename]
List directory with certain files
\item[DEL filename]
Delete file 
\item[FMS]
Format in single density
\item[FME]
Format in enhanced density
\item[FMD]
Format in double density, requires special or enhanced disk drive
\item[LOC filename]
Lock file
\item[UNL filename]
Unlock file
\item[REN filename,newname]
Rename file
\item[LOA filename]
Load object file
\item[LOA filename/N]
Display object file load address(es) but do not load
\item[LOA filename,start]
Load raw data file to address \fpara{start}, ignore COM header
\item[SAV filename,start,end]
Save object file with the memory content from address \fpara{start} to
(and including) \fpara{end}
\item[SAV filename/N,start,end]
Save file without COM header (raw file) with the memory content from address \fpara{start} to
(and including) \fpara{end}
\end{fcmdlist}

\subsection{Characteristics of specific commands}
Some commands have specific characteristics compared to standard DOS commands.
These characteristics are either improvements or are the logical result of
working in a freezer environment.

While renaming files with  \fcmd{REN} command, the specification of the new file
name may contain arbitrary wildcards. The wildcard characters are replaced by
the characters from the original file name. This allows for working on groups of
files efficiently, not being restricted to the primary or secondary name (i.e.
the extension) of the file.

Object files can be loaded using \fcmd{LOA}. The memory area to which the file
is loaded is also displayed. In case of compound files, all memory areas are
displayed. The complete 64k memory area can be used as target.  This includes
loading directly into the hardware registers, which are currently frozen. The
loaded program will \textbf{not} be started, to prevent conflicts with the
memory management logic of the freezer in case multiple segments are loaded.

Adding the option \fcmd{/N} to the load command will display the memory area
into which the program would be loaded but will not actually load the data into
memory. This is useful in case you are only interested in the memory location to
which an object file would be loaded, \ie

\fcmd{LOA FONT.COM/N}

The command \fcmd{LOA} can also be used to load raw data (without COM header).
In this case the address to which the data shall be loaded must be specified, \ie

\fcmd{LOA FONT.DAT,8000}

\subsection{Error messages}
When using the DOS and the load and save functions of the freezer for frozen
programs, errors can occur. The corresponding error messages are displayed 
as plain English text.

\begin{fmsglist}
\item[FILE NOT FOUND]
The file could not be found
\item[FILE\# MISMATCH]
The (internal) file number does not match, the file structure is likely
to be damaged. Rescue the other files if possible and format the disk
\item[BAD DISK I/0]
The command cannot be executed due to a bus or disk error or because
the write protection is active
\item[NO DRIVE]
The disk drive does not respond
\item[DISK FULL]
The disk is full
\item[FILE LOCKED]
The file is locked
\item[DIRECTORY FULL]
The directory is full (at most 64 files per disk)
\end{fmsglist}
