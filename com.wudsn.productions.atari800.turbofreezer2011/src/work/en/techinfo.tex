\chapter{For experts: Technical details}

\section{Hardware}
The heart of the \frz is the Xilinx XC95144XL CPLD. It contains the complete
logic of the freezer and controls the freezer RAM as well as the cartridge
emulation. There are 1MB flash ROM and 1MB battery backed freezer RAM present
which are shared between the functional blocks, i.e. freezer, cartridge
emulation and 512k RAM extension.

When the freezer RAM or the flash ROM of the freezer have to be mirrored into
the Atari, the internal RAM and ROM of the Atari must be deactivated. This is
achieved using a small trick. The freezer pulls the line of the ANTIC refresh
pin, which is intended to be an output, to \fq{low}. This makes the MMU in
the Atari believe that ANTIC is performing a memory refresh, hence it
deactivates all built-in memory.
According to Bernhard Engl, who works as a full-time chip designer nowadays, it
is not dangerous to pull a \fq{high} level \fq{low} from outside in the
NMOS-Depletion-Load technology, because that is how NMOS works anyway.
If the ANTIC had been manufactured in CMOS technology, this approach would have
been dangerous and could have damaged the ANTIC -- and there never had been a
\frz.

\section{Memory partitions}

\subsection{Freezer RAM}
512k of the freezer RAM are directly assigned to the 512k RAM extension.
The remaining 512k RAM are shared between the freezer and the cartridge
emulation. The freezer can access the upper 128k (banks \fdecr{48}{63}) of this
area. The topmost 16k (banks 62 and 63) are exclusively reserved for the freezer
(\ie for the hardware register shadowing). The bank switching in the freezer
addresses the freezer RAM not in 8k banks, like the cartridge emulation, but in
4k banks. The cartridge emulation can access the complete RAM except for the
last 16k, making available a maximum of 496k (banks \fdecr{0}{61}).

\subsection{Flash ROM}
The cartridge emulation can access the complete 1MB flash ROM, the freezer can
access the upper 512k thereof. The topmost 64k are reserved for the freezer
software, the Oldrunner OS and the cartridge emulation menu. The topmost 64k
(banks \fdecr{120}{127}) are currently used as follows.

\clearpage

\begin{itemize*}
\item Bank 120: Free
\item Bank 121: Free
\item Bank 122: Free
\item Bank 123: Freezer software bank 3
\item Bank 124: Freezer software bank 1 (start bank)
\item Bank 125: Freezer software bank 2
\item Bank 126: Cartridge emulation menu
\item Bank 127: Oldrunner OS
\end{itemize*}

The banks 127 (Oldrunner OS), 126 (cartridge emulation) and 124 (freezer
software) are \fq{hardwired} in the CPLD logic. The usage of the remaining
banks \fdecr{120}{123} and 125 may change in later versions.

\section{Freezer}
As long as the \fsw{Freeze} button is not pressed, the freezer is completely
invisible for the Atari. That means, no software can detect if there is a
freezer attached to the Atari or not. Only when the button is pressed, the
freezer becomes visible and active.

\subsection{Shadowing of hardware registers}

The freezer uses the following \fq{trick} to read the content of the
non-readable hardware registers of ANTIC, GTIA and POKEY. Whenever there is a
write access to the area \fhexr{D000}{D7FF}, the freezer RAM is also activated.
This way the freezer software can read the data of the custom chips from there
without problems. The write access to the RAM is only performed if the
\fsw{Freeze} button has not been pressed yet. Otherwise the freezer software
itself would overwrite the stored data.

The addresses of custom chips in the Atari are not completely decoded. The ANTIC
can for example be accessed via the addresses \fhex{D40x}, \fhex{D41x}, \dots.
When \fhex{22} is written to \fhex{D400} and \fhex{00} is written to \fhex{D410}
afterwards, both values will end up in the \freg{DMACTL} register of the ANTIC.
\freg{DMACTL} will have the value \fhex{00} after the 2nd write access.

Some programs use these tricks. To enable the freezer to restore the exact state
upon resuming, the freezer logic has to take this in account. Therefore the
non-decoded address lines are pulled \fq{low} when shadowing the hardware
register. For ANTIC (\fhex{D4xx}) and POKEY (\fhex{D2xx}) these are A4--A7,
for GTIA (\fhex{D0xx}) these are A5--A7.

If the Atari contains stereo POKEY extension, then the first POKEY is accessible
via \fhex{D20x}, \fhex{D22x}, \dots, the second POKEY via \fhex{D21x},
\fhex{D23x}, \dots . Turning the \fsw{Stereo} switch of the freezer right (\fval{ON})
changes the shadowing logic. During access to \fhex{D2xx} the address lines
A5--A7 are pulled \fq{low} then and the data of both POKEYs are stored
correctly.

\clearpage

\subsection{Mirroring the freezer memory in}

When the freezer is activated, the freezer RAM resp. the flash ROM
is mirrored to the area \fhexr{0000}{3FFF}. Bank switching is used to choose
between 32 RAM banks of 4k each (a total of 128k) and 64 ROM banks of 8k each
(a total of 512k). The memory map in the Atari looks as follows.

\begin{itemize*}
\item \fhexr{0000}{0FFF}: Fixed 4k RAM bank (bank 31)
\item \fhexr{1000}{1FFF}: Switchable 4k RAM bank (bank \fdecr{0}{31})
\item \fhexr{2000}{3FFF}: Switchable 8k ROM bank (bank \fdecr{64}{127})
\end{itemize*}

The 64 ROM banks of 8k each are located at the end of the flash ROM and
correspond to the banks \fdecr{64}{127} of the cartridge emulation. The 32 RAM
banks of 4k each are located at the end of the freezer RAMs and correspond to
the banks \fdecr{48}{63} of the cartridge emulation.
The last 4k bank (bank 31) is used for the shadowing of the hardware registers
and is mirrored to the address \fhex{0000} in the Atari if the freezer is
activated.

\subsection{Activating the freezer}

The freezer logic has 5 internal states.

\begin{itemize*}
\item Freezer deactivated
\item Freezer half activated
\item Freezer activation triggered
\item Freezer activated
\item Freezer temporarily deactivated
\end{itemize*}

After switching the Atari on, the freezer is in the state \fq{deactivated}.
In this state and in the state \fq{half activated} all write access to the
custom chip area is stored in the shadow RAM.

When the \fsw{Freeze} button is pressed, nothing happens at first. Only when the
first access to the area \fhexr{FFF8}{FFFF} (which contains also the IRQ and NMI
vectors) is performed and the \fsw{Freeze} button is still pressed, the freezer
reacts. It saves the state of the A2 line (which is required later on to
distinguish between IRQ and NMI) and enters the state \fq{half activated}. In
the next clock cycle there are now these two possibilities:

\begin{itemize*}
\item If the \fsw{Freeze} button is no longer pressed or an access to
another memory area than \fhexr{FFF8}{FFFF} is performed, the freezer enters
the \fq{deactivated} state again.
\item If the \fsw{Freeze} button is still pressed and there is an access to
the memory area \fhexr{FFF8}{FFFF} again, the freezer performs the transition to
the state \fq{activation triggered} as described below.
\end{itemize*}

The freezer deactivates the memory in the Atari and puts the value \fhex{21}
on the data bus. This changes the interrupt vectors to the address \fhex{21xx},
as the CPU is currently trying to load the high byte of the interrupt vector.
The low byte stems from the previous access to the \fhexr{FFF8}{FFFF}
area and therefore has the original value.

After this, the freezer enters the state \fq{activation triggered}.
Just like in the state \fq{activated} the \frz now mirrors its own memory in.
The ROM bank 124 with the start code of the freezer software is mirrored to the
memory area \fhexr{2000}{3FFF}. Starting at \fhex{2000} a complete page full of
\fq{RTI} instructions is present. After this, \ie starting at \fhex{2100} a
so-called \fq{NOP slide}, \ie a page full of \fq{NOP} instructions for the
\fq{rerouted} interrupt routine is present. The actual freezer software starting
with the initialization routine follows after that.

During the activation phase, the occurrence of IRQs is no problem, because they
are masked by the CPU during the execution of an interrupt routine anyway. If an
NMI occurs during the activation phase, the freezer reroutes its interrupt
vectors to the address \fhex{20xx}. There only \fq{RTI} are present and the NMI
will be ended again immediately. This means the freezer \fq{suppresses} all further
interrupts during the activation phase.

After running through the NOP slide and disabling the interrupts the software
puts the freezer into the state \fq{activated} via a write access to the address
\fhex{D720}. If the now activated freezer software wants to access the Atari
memory in the area \fhexr{0000}{3FFF}it first has to disable the freezer RAM
resp. the flash ROM temporarily. This is done via a write access to the address
\fhex{D700}, which puts the freezer into the state \fq{temporarily deactivated}.
An additional write access to the address \fhex{D700} puts the freezer back into
the state \fq{activated} and the freezer RAM resp. the flash ROM are mirrored in
again.

When the freezer software is exiting and the interrupted program is resumed, the
freezer has of course to be put into the state \fq{deactivated} again.
This is done via a read access to the address \fhex{D700}.

\subsection{Configuration registers}

When the freezer is activated, a number of configuration registers is mirrored
into the area \fhexr{D700}{D7FF}. When these registers are accessed, the
internal memory in the Atari is deactivated to prevent problems with potentially
installed internal extensions.

\begin{fadrdef}{\fhexr{D700}{D70F}}{Freezer State Control}{w}
- & - & - & - & - & - & - & -
\end{fadrdef}
\noindent Write access to this register toggles the freezer between the states
\fq{activated} and \fq{temporarily deactivated}.

\begin{fadrdef}{\fhexr{D700}{D70F}}{Freezer Disable}{r}
- & - & - & - & - & - & - & -
\end{fadrdef}
\noindent A read access to this register put the freezer into the state
\fq{deactivated}.

\clearpage

\begin{fadrdef}{\fhexr{D710}{D71F}}{Freezer RAM A4/A5}{r/w}
- & - & - & - & - & - & - & -
\end{fadrdef}
\noindent A read access to this register changes the handling of the address
lines A4 and A5 when accessing the freezer RAM (\fhexr{0000}{1FFF}). A4 is set to the state A2 had
when the freezer was activated. A5 is set to the current state of the
\fsw{Stereo} switch. This way the freezer software can detect if the freezer was
activated via an NMI or via an IRQ and if the stereo POKEY mode is active or
not. A write access to this register resets the logic to the normal handling
again.

\begin{fadrdef}{\fhexr{D720}{D72F}}{Freezer Startup Complete}{r/w}
- & - & - & - & - & - & - & -
\end{fadrdef}
\noindent A read or write access to this register puts the freezer from the
state \fq{activation triggered} into the state \fq{activated}.

\begin{fadrdef}{\fhexr{D740}{D77F}}{Flash ROM Bank Select}{r/w}
- & - & - & - & - & - & - & -
\end{fadrdef}
\noindent A read or write access to these registers selects the 8k flash ROM
bank which is mirrored to \fhexr{2000}{3FFF}. An access to \fhex{D740} selects
bank 64, \fhex{D741} selects bank 65, \dots \fhex{D77F} selects bank 127.

\begin{fadrdef}{\fhexr{D780}{D79F}}{Freezer RAM Bank Select}{r/w}
- & - & - & - & - & - & - & -
\end{fadrdef}
\noindent A read or write access to these registers selects the 4k freezer RAM
bank which is mirrored to \fhexr{1000}{2FFF}. An access to \fhex{D780} selects
bank 0, \fhex{D781} selects bank 1, \dots \fhex{D79F} selects bank 31.

\subsection{Freezer software and OS ROMs}

The freezer software uses its own code for the most functions (\eg the keyboard
input) instead of using the routines of the operating system. This makes the
software mostly independent of the used operating system and also independent of
the state of the Atari when freezing is triggered. Yet the used operation system
must provide some functions to ensure correct operation of the freezer software
and these functions must behave like in the original operating system.

\begin{itemize*}
\item Standard character set at \fhexr{E000}{E3FF}
\item NMI and IRQ handling
\item SIO function (\fhex{E459}), used by the \fcmd{SIO} command in the
debugger and when save to or loading from cassette
\end{itemize*}

All alternative OS versions which have been available so far, like for example
\fq{QMEG OS} or \fq{Romdisk OS}, fulfill these requirements. Problems came up
recently with the new \fq{Ultimate~1MB BIOS}. It neither contains a standard
interrupt handling nor a character set at \fhex{E000}. Therefore the screen is
completely garbled when the freezer is activated from within the
\fq{Ultimate~1MB BIOS}.

\section{512k RAM extension}
A separate 512k RAM chip is included for the 512k RAM extension.
This way the freezer, the cartridge emulation with modules in the freezer RAM,
and the RAM extension can be used in parallel to the full extent.

Because the PIA signals are not available on the PBI, the CPLD must mimic the
behavior of the PIA registers. The PIA in the Atari keeps working \fq{in
parallel}, so BASIC, OS ROM, and Self Test are controlled as usual.
Read access to the PIA is ignored by the CPLD, \ie in this case the Atari reads
the real PIA registers as usual. The logic only reacts on write access to the
addresses \fhex{D301} (\freg{PORTB}) and \fhex{D303} (\freg{PBCTL}). The CPLD
also saves the value of bit 2 of \freg{PBCTL}, which controls if the access is
performed on the data register \freg{PORTB} (bit 2~=~1) or on the data direction
register (bit 2~=~0).

The CPLD stores the data of a write access to \fhex{D301} either in the
internal data register or in the internal data direction register. If a PIA pin
it setup as an input, then pull-up resistors in the Atari pull it \fq{high}. If
a PIA pin is setup as an output, then the data register controls if a \fq{low}
or a \fq{high} signal is output.

The CPLD emulates exactly this behavior and therefore behaves exactly like
every other RAM extension in the Atari. The Oldrunner OS for example sets
\freg{PORTB} up as input because the 400/800 had 4 joystick ports -- but no
RAM/ROM control via \freg{PORTB}. This automatically deactivates every RAM
extension, because bit 4 of \freg{PORTB} must be \fq{low} to activate the RAM
extension. But the emulated pull-up resistor in the CPLD pulls the pin \fq{high}
automatically in this case, like it would have been the case in the PIA.

The rest of the RAM extension logic is relatively simple. If the \fsw{Ramdisk}
switch is turned \fval{ON} and if bit 4 of the emulated \freg{PORTB} register
has the value \fval{0}, then the internal memory of the Atari is disabled for
every access to \fhexr{4000}{7FFF} and a 16k bank of the 512k RAM extension is
mirrored in instead. The bits 2,3,5,6,7 of the emulated \freg{PORTB} register
determine in this case which of the 32 banks of 16k size shall be used.


\section{Oldrunner}
The Oldrunner is active if the \fsw{OldOS} switch is turned to \fval{ON}).
For every access to \fhexr{E000}{FFFF} bank 127 of the flash ROM is mirrored in and for every
access to \fhexr{C000}{CFFF} the internal memory is deactivated. This makes the
Atari behave exactly like the original Atari 400/800, which did not have any
memory at \fhexr{C000}{CFFF} . When the Math Pack area (\fhexr{D800}{DFFF}) is
accessed, the internal OS ROM of the Atari will be mirrored in and the Math Pack
contained in the OS will be used.

\section{Cartridge emulation}

The cartridge emulation offers very many configuration options. Controlling all
these options only with switches would not have been handy. Therefore the major
part of the configuration is performed via software, for example from within the
menu of the cartridge emulation or from within the debugger of the freezer.

\subsection{Configuration switches}

One function of the \fsw{CartEmu} and \fsw{FlashWrite} switches is to control
access to the configuration registers. If both switches are turned \fval{OFF},
then the configuration registers are inaccessible. If at least one of the
switches is turned \fval{ON}, then the configuration registers
\fhexr{D590}{D596} can be accessed.

If the \fsw{CartEmu} switch is turned \fval{ON} when the Atari is switched on,
then bit 2 of the register \fhex{D595} is initialized to \fval{1}. This disables the
rest of the cartridge emulation configuration and activates the cartridge
emulation menu.

If the \fsw{FlashWrite} switch is turned \fval{OFF}, then no write access to the
flash ROM resp. the freezer RAM is possible. The only exception is the 
\fval{8k+RAM} mode in which write access to the optional 8k freezer RAM bank at
\fhexr{8000}{9FFF} is always possible. If the \fsw{FlashWrite} is turned
\fval{ON}, then write access to the flash ROM and the freezer RAM is possible
after is has been enabled via bit 0 of the register \fhex{D595}.

\subsection{Configuration registers}

The configuration registers at \fhexr{D590}{D596} become accessible if at least
one of the \fsw{CartEmu} or \fsw{Flash Write} switches is turned \fval{ON}.
These registers are writeable and readable. All unused bits of these registers
should be set to \fval{0}. During a read access to the registers, all unused bits also
return \fval{0}. All registers except register \fhex{D595} are set to \fhex{00}
during power-on. If the \fsw{CartEmu} switch is turned \fval{OFF}, the register is set
to \fhex{00}. If it is turned \fval{ON}, the register is set to \fhex{04}.
Depending on the active modules type (\eg in OSS or AtariMax mode) additional
registers are accessible in the area \fhexr{D500}{D5FF}.

\begin{fadrdef}{\fhex{D590}}{Bank}{r/w}
- & \multicolumn{7}{c|}{\fcmd{bank}}
\end{fadrdef}
\noindent Number of the selected bank for the cartridge emulation (\fdecr{0}{127}). Bit 0 is
ignored in 16k mode.

\begin{fadrdef}{\fhex{D591}}{Bank Enable}{r/w}
- & - & - & - & - & - & - & \fcmd{enable}
\end{fadrdef}
\noindent State of the ROM bank of the module. The value \fval{0} disables the
module, \fval{1} enables it. In \fval{8k+RAM} mode this bit only controls the
area at \fhexr{A000}{BFFF}. The optional RAM area at \fhexr{8000}{9FFF} is
controlled by the register \fhex{D593}.

\begin{fadrdef}{\fhex{D592}}{RAM Bank}{r/w}
- & - & \multicolumn{6}{c|}{\fcmd{bank}}
\end{fadrdef}
\noindent Number of the selected RAM bank (\fdecr{0}{63}) in \fval{8k+RAM} mode.

\begin{fadrdef}{\fhex{D593}}{RAM Bank Enable}{r/w}
- & - & - & - & - & - & - & \fcmd{enable}
\end{fadrdef}
\noindent State of the RAM bank of the module. In \fval{8k+RAM} mode the value
\fval{0} disables the RAM bank, \fval{1} enables it.

\begin{fadrdef}{\fhex{D594}}{Mode}{r/w}
- & - & - & - & - & \multicolumn{3}{c|}{\fcmd{mode}}
\end{fadrdef}
\noindent Type of the emulated module. The following modules types are
supported.
\begin{itemize*}
\item \fval{0}: Cartridge emulation deactivated
\item \fval{1}: 8k module
\item \fval{2}: 8k+RAM module
\item \fval{3}: 8k module with \frz 2005 bank switching
\item \fval{4}: 16k module
\item \fval{5}: OSS module
\item \fval{6}: AtariMax module
\item \fval{7}: Reserved, do not use!
\end{itemize*}

\begin{fadrdef}{\fhex{D595}}{Misc Configuration}{r/w}
- & - & - & - & - & \fcmd{menu} & \fcmd{source} & \fcmd{write}
\end{fadrdef}
\begin{fcmdlist}	
\item[menu] Cartridge emulation menu control 
  \begin{fvallistn}
  \item[0] Normal cartridge emulation
  \item[1] Activate cartridge emulation menu (bank 126)
  \end{fvallistn}
\item[source] Module data source
  \begin{fvallistn}
  \item[0] Flash ROM
  \item[1] Freezer RAM
  \end{fvallistn}
\item[write] Write access control
  \begin{fvallistn}
  \item[0] Write access disabled
  \item[1] Write access allowed, if the \fsw{FlashWrite} switch is turned
  \fval{ON} in addition
  \end{fvallistn}
\end{fcmdlist}

\clearpage

\begin{fadrdef}{\fhex{D596}}{SDX Configuration}{r/w}
- & - & - & - & - & - & \multicolumn{2}{c|}{\fcmd{config}}
\end{fadrdef}

\begin{fvallist}
\item[00] SpartaDOS~X deactivated
\item[01] SpartaDOS~X in freezer RAM starting at bank 0
\item[10] SpartaDOS~X in flash ROM starting at bank 0
\item[11] SpartaDOS~X in flash ROM starting at bank 64
\end{fvallist}

\subsubsection{Additional registers in 8k mode with \frz 2005 bank switching}

\begin{fadrdef}{\fhexr{D540}{D57F}}{Bank Select}{r/w}
- & - & - & - & - & - & - & -
\end{fadrdef}
\noindent A read or write access to this area selects the bank. An access to
address \fhex{D540} selects bank 0, \fhex{D541} selects bank 1, \dots,
\fhex{D57F} selects bank 63. If the bank register \fhex{D590} is set to a bank
in the range \fdecr{64}{127}, then \fhexr{D540}{D57F} will refer to the banks
\fdecr{64}{127} instead of the banks \fdecr{0}{63}.

\begin{fadrdef}{\fhex{D580}}{Cart Disable}{r/w}
- & - & - & - & - & - & - & -
\end{fadrdef}
\noindent A read or write access to this register disables the module.

\begin{fadrdef}{\fhex{D581}}{Cart Enable}{r/w}
- & - & - & - & - & - & - & -
\end{fadrdef}
\noindent A read or write access to this register enables the module.

\begin{fadrdef}{\fhex{D584}}{Write Disable}{r/w}
- & - & - & - & - & - & - & -
\end{fadrdef}
\noindent A read or write access to this register disables write access to the
module.

\begin{fadrdef}{\fhex{D585}}{Write Enable}{r/w}
- & - & - & - & - & - & - & -
\end{fadrdef}
\noindent A read or write access to this register enables write access to the
module.

\clearpage

\subsubsection{Additional registers in OSS mode}

\begin{fadrdef}{\fhexr{D500}{D50F}}{OSS Bank Select}{r/w}
- & - & - & - & - & - & - & -
\end{fadrdef}
\noindent These registers implement the bank select and cart disable as in OSS
modules of type \fq{M091}. A read or write access to the respective address
activates the following configuration.

\begin{flist}
\item[\fhex{D500}] Module enabled, \fhexr{A000}{AFFF} bank 1,
\fhexr{B000}{BFFF} bank 0
\item[\fhex{D501}] Module enabled, \fhexr{A000}{AFFF} bank 3,
\fhexr{B000}{BFFF} bank 0
\item[\fhex{D508}] Module disabled
\item[\fhex{D509}] Module enabled, \fhexr{A000}{AFFF} bank 2,
\fhexr{B000}{BFFF} bank 0
\end{flist}

\subsubsection{Additional registers in AtariMax mode}

\begin{fadrdef}{\fhexr{D500}{D57F}}{Bank Select}{r/w}
- & - & - & - & - & - & - & -
\end{fadrdef}
\noindent A read or write access to this area activates the module and selects
the bank. An access to address \fhex{D500} selects bank 0,
\fhex{D501} selects bank 1, \dots, \fhex{D57F} selects bank 127.

\begin{fadrdef}{\fhex{D580}}{Cart Disable}{r/w}
- & - & - & - & - & - & - & -
\end{fadrdef}
\noindent A read or write access to this register disables the module.

\textbf{Caution:} In the original AtariMax module the addresses
\fhexr{D581}{D5FF} can also be used to disable the module. The cartridge
emulation only offers this single address for this.


\clearpage

\subsubsection{Additional registers when SpartaDOS~X is active}

\begin{fadrdef}{\fhex{D5E0}}{Bank Select}{r/w}
\multicolumn{2}{|c|}{\fcmd{control}} & \multicolumn{6}{c|}{\fcmd{bank}}
\end{fadrdef}

\begin{fcmdlist}
\item[control] SpartaDOS~X control
  \begin{fvallistn}
  \item[00] SpartaDOS~X enabled, additional module disabled
  \item[01] SpartaDOS~X enabled, additional module disabled
  \item[10] SpartaDOS~X disabled, additional module enabled
  \item[11] SpartaDOS~X disabled, additional module disabled
  \end{fvallistn}
\item[bank] SpartaDOS~X bank select (bank \fdecr{0}{63})
\end{fcmdlist}

If the SpartaDOS~X emulation has been activated via the register \fhex{D596}, it
has priority over the rest of the cartridge emulation. The cartridge emulation then behaves
as if an original SpartaDOS~X module was \fq{plugged in}.
