\section{Debugger}

A debugger is used by machine language programmers to display, alter and improve
object code and other memory contents directly in the computer memory. Doing
this in the environment of a freezer brings certain advantages. Because of the way
the freezer works, there are also some limitations that may be circumvented by
using the right procedures.

\subsection{Access to memory and hardware registers}

Without a doubt, the biggest advantage is the possibility to work with
the frozen system state. Doing this creates the impression of
having a second computer, that's linked into the first, stopped
Atari. It enables you to \fq{look into} and change the stopped Atari
and to resume any time. A debugger without freezer always has severe
problems because of its own memory usage and because the I/O operations of
the debugger itself change the system state. A programmers' term for this is
\fq{trashing}, meaning turning the content of the system into \fq{trash}.

A standard debugger requires RAM for its own operation and will
trash the data that was put there by the program that's being looked at.
And even if the debugger is more sophisticated (and comes with its own RAM),
it will still trash the hardware registers of ANTIC, POKEY and GTIA. There are
no OS shadow registers for player missile graphics and sound registers and the
existing OS shadow register will be deactivated by many programs anyway.
As a result of the trashed hardware registers, it will be a lot of work, if even
possible, to resume a program. And apart from this, debuggers without a
freezer don't have the ability to display the content of the non-readable
(write-only) hardware registers. So even if the register contents are
not trashed by the debugger, the values in the registers remain unknown.

Using the built-in debugger of the \frz, all information about the system state is
available. The contents of the hardware registers (i.e. what has been written
there, not the status returned by a read operation) can be viewed in the I/O
area and may be changed, without having to fear a system crash. The changes are
applied only to the frozen program. You can work with the complete RAM area
(even the RAM under the OS ROM), without trashing or fearing a crash.

In addition, there are various advantages resulting from running in the
environment of the \frz. The load and save functions and the DOS functions allow
for instantaneous testing and retrying, without lengthily reloading a program.
And if the change works not as expected, it may be undone instantly without
problems. Using this, even horrendous code monsters that overwrite parts of the
DOS and that cannot be processed in a simple way, are not frightening anymore.

The software of the original \frz from 1987 had some disadvantages, among
other reasons due to size constraints. Changes were always only applied
to the frozen system state and changes to the hardware registers were only
visible after resuming. This made it hard for example to work with the RAM
extension.

The extensions of the software that even been implemented over time have
meanwhile removed the majority of the original constraints. Using the direct I/O
mode, the debugger can access extensions in the custom chip area and bank
switching modules directly. And the \fcmd{PB} command allows direct control of
the access to the RAM extension, the OS ROM and the RAM under the OS ROM.

\subsection{Command summary}
The set of commands is kept minimal and is aligned with the Atari \fq{Editor
Assembler Cartridge} which makes it similar to many other debuggers. Hence a
tabular summary of the command should suffice.
The commands must not contain any spaces and must start immediately behind the
prompt. Only the command line can be used for input. All changes are logged in
the output area. All entries are in hexadecimal notation.
By omitting a value, memory locations or registers may be left unchanged.

Example: \fcmd{C100{\textless}0A,{},4D} will change the content of the memory
location \fhex{100} to \fhex{0A} and of the memory location \fhex{102} to
\fhex{4D}. After this, the internal address counter will point to \fhex{103}.
The content of the memory location \fhex{101} remains unchanged.

For all commands, the address may be omitted and the debugger will use the
internal address counter or an end address that will yield a reasonable output.
The output can be stopped at any time by pressing \fkey{S} and resumed by
pressing \fkey{Q}.

\begin{fcmdlist}
\item[Q] 
Go to freezer main menu
\item[D start]
Display 8 bytes plus ATASCII characters starting at address \fpara{start} 
\item[D start,]
Display 128 bytes starting at address \fpara{start}
\item[D start,end]
Display memory content from address \fpara{start} to \fpara{end}
\item[I start]
Display 8 bytes plus characters in internal Atari screen code starting at address \fpara{start} 
\item[I start,]
Display 128 bytes starting at address \fpara{start}
\item[I start,end]
Display memory content from address \fpara{start} to \fpara{end}
\item[L start]
Disassemble starting at address \fpara{start}, display one screen page
\item[L start,end]
Disassemble from address \fpara{start} to \fpara{end}
\item[DL start]
Display display list starting at address \fpara{start}
\item[DL start,end]
Display display list from address \fpara{start} to \fpara{end}
\item[C start{\textless}byte1,byte2{\dots}]
Change memory content starting at address \fpara{start}
\item[VEC]
Display OS vectors
\item[HAT]
Display handler table
\item[M]
Display memory usage map
\item[PB]
Display content of the memory management register \freg{PORTB} (controls
access to RAM extension, OS, BASIC)
\item[PB{\textless}value]
Set content of the  memory management register \freg{PORTB} to \fpara{value} 
\item[DIO]
Display direct I/O mode
\item[DIO{\textless}value]
Activate/deactivate direct I/O mode, use with caution
\item[R]
Display registers 
\item[R{\textless}value1,value2\dots]
Change registers
\item[G start]
Set return address for resuming to \fpara{start}
\item[/start,end/value\dots]
Search for \fpara{value\dots} in the memory area from address
\fpara{start}{\dots}\fpara{end}
\item[BM start,end,target]
Move the memory area from \fpara{start{\dots}end} to\newline
\fpara{target{\dots}target$+$(end$-$start)}
\item[BS start,end,value]
Set memory area from \fpara{start{\dots}end} to \fpara{value}
\item[BC start,end,start2]
Compare memory area from \fpara{start{\dots}end} to the memory area starting at\linebreak
\fpara{start2}
\item[PR{\textless}value]
Activate/deactivate printer output
\item[; TEXT]
Print a text or comment
\item[SR number]
Read sector \fpara{number}
\item[SW number]
Write sector \fpara{number}
\item[SIO \dots]
Execute SIO command
\item[SIOR]
Reset high speed SIO routine
\item[V]
Display interrupt and content of the register \freg{VCOUNT} at the time of the
freeze
\item[V{\textless}value]
Set the content of the register \freg{VCOUNT} for the time when program will be
resumed
\item[a COMMAND\dots]
AtariSIO remote control
\item[VER]
Display freezer software version
\end{fcmdlist}

Remark: With the exception of the spaces immediately following the
command, all spaces in the above list are only there for readability and must
not be entered with the command. The program counter (PC) cannot be changed
using \fcmd{R{\textless}} due to its special handling. Use the \fcmd{G} command to
change its value.

\subsection{Input of values (hexadecimal, decimal, internal code)}
Addresses and byte values are input in hexadecimal notation by default.
Use the percent sign \fq{\fcmd{\%}} as a prefix to enter an address or byte value as
decimal number. The following command corresponds to the statement
\fq{POKE~710,10} in BASIC.
\begin{fcode}
C%710<%10
\end{fcode} 
Byte values can also be given in ATASCII or internal Atari screen code. For
ATASCII, the single quote \fq{\fcmd{'}} must be used as prefix, for screen code
the at sign \fq{\fcmd{@}} must be used.
\begin{fcode}
C0600<'H,'a,'l,'l,'o
C9C40<@S,@c,@r,@e,@e,@n
\end{fcode}

\subsection{Input of vectors}
For working with vectors (\ie display list or interrupt vectors) an additional
variant for specifying addresses is available. If the prefix \fq{\fcmd{*}} is used
for an address, the 2 byte vector at that address is evaluated and the content
of the vector is used as effective address. This can be used for example to
display the current display list very effectively.
\begin{fcode}
DL*230
DL*%560
\end{fcode}

\subsection{Access to extended memory, RAM, ROM}
The debugger offers full control over the memory management functions of the
Atari XL/XE when accessing the memory content. The memory management unit is
controlled by the register \freg{PORTB} (\fhex{D301}) of the PIA.
This way it is possible to read and change the content of the RAM extension
(\fhexr{4000}{7FFF}) or the RAM under the OS ROM
(\fhexr{C000}{CFFF},\fhexr{D800}{FFFF}) without problems.

The \freg{PORTB} control only refers to the access from within the debugger.
Changes do not affect the frozen program. To change the state after resuming, the
frozen content of \freg{PORTB} must be changed using \fcmd{C~D301{\textless}byte}.

When the freezer is activated, the current value of \freg{PORTB} is used as
default for the debugger. That means you see the same state in the debugger as
the frozen program would see. The \freg{PORTB} control can be displayed and
changed with the \fcmd{PB} command:
\begin{fcmdlist}
\item[PB] Display current \freg{PORTB} value
\item[PB{\textless}value] Set new \freg{PORTB} value
\end{fcmdlist}

If you would like to access the RAM under the OS ROM, enter
\fcmd{PB{\textless}FE} to set bit 0 of \freg{PORTB} to 0. With
\fcmd{PB{\textless}E3} access to the first bank of the RAM extension in Atari
130~XE is activated. The content of \freg{PORTB} at the time of the freeze can
be displayed using \fcmd{D~D301} -- provided it was not changed already using
\fcmd{C~D301{\textless}value}.

The access to the frozen PIA registers (\fhex{D3xx}) is subject to a special
handling in the freezer. Normally, the values of the PIA registers are repeated
every 4 bytes in the Atari. In the debugger, 8 bytes are displayed instead.
\begin{flist}
\item[\fhex{D300}, \fhex{D301}]
are the normal registers \freg{PORTA} and \freg{PORTB}
\item[\fhex{D302}, \fhex{D303}]
contain the data direction registers for \freg{PORTA} and \freg{PORTB}.
They are normally accessible via \freg{PORTA} resp. \freg{PORTB} if bit 2
of \freg{PACTL} / \freg{PBCTL} is set to 0.
\item[\fhex{D304}, \fhex{D305}]
contain the values of \freg{PACTL} resp. \freg{PBCTL} which are normally
accessible via the address\fhex{D302} resp. \fhex{D303}.
\item[\fhex{D306}, \fhex{D307}]
are unused
\end{flist}

\subsection{Display display list}
The \fcmd{DL} command prints the ANTIC menmonics as described in the ANTIC data
sheet, starting at the specified memory address.
\begin{fcmdlist}
\item[BLK x] x blank lines
\item[CHR x] Text mode x
\item[MAP x] Bitmap mode x
\item[JMP adr] Jump to given address
\item[JVB adr] Wait for vertical blank, then jump to given address
\end{fcmdlist}
Right to the mnemonic the following options are printed, if present:
\begin{fcmdlist}
\item[LMS adr] Load memory scan counter (pointer to the current screen memory
address)
\item[H] Horizontal scrolling enabled
\item[V] Vertical scrolling enabled
\item[I] Trigger display list interrupt (DLI)
\end{fcmdlist}

\subsection{Search in memory}
The search command can be used in various manners. Use
\fcmd{/start/value1,value2\dots} to search the specified byte sequence starting
at the specified start address. At the first occurrence of the byte sequence,
the search stops and prints the address. The internal address counter is set to
the found address, so you can disassemble from that address immediately using
the \fcmd{L} command. The search can be resumed with the \fcmd{/} command.

The byte sequence can consist of up to 8 bytes, which is more than sufficient
in most cases. It's also possible to omit bytes from the sequence. These bytes
are ignored by the search.
\fcmd{/1000/8D,{},D4} will find the first address starting from \fhex{1000} that
contains the command \fq{STA~\fhex{D4xx}} to write a byte into the ANTIC.

The byte sequence can also contain a bit mask. To do this, the separator
\fcmd{\&} followed by a bit mask must be appended to the byte, like for example
\fcmd{03\&0F}. The content of the memory will combined with the bit mask using
AND and will then be compared with the byte value from the byte sequence.

If an end address is specified in addition to the start address, all addresses
where the byte sequence is found are printed. That means the search does not
stop at the first occurrence when entering for example
\fcmd{/start,end/value1,value2\dots}.
Many times, the complete memory is to be examined. Therefore there is the very short
variant \fcmd{//value1,value2\dots} which is identical to
\fcmd{/0000,FFFF/value1,value2\dots}.

If the byte sequence is long or the memory area is large, the search may take
several seconds. Using the \fkey{BREAK} key, the search can be interrupted at
any time. It can then be resumed with the \fcmd{/} command.

\subsection{Execute SIO commands}

The commands \fcmd{SW}, \fcmd{SR}, \fcmd{SIO} as well as the built-in DOS use an
internal sector buffer. The memory manager mirrors it to the address \fhex{D700}
in the frozen address space, where it can be edited by the debugger.
Physically it is not present at that address. To read a sector from a different
drive than \fcmd{D1:} or to write it there, just prepend the sector number with
\fcmd{Dx:}.
\begin{fcode}
SR 100
SR D2:200
SW D3:300
\end{fcode}

With the \fcmd{SIO} command, arbitrary commands can be sent to the SIO, just
like using the SIO vector \fhex{E459}. The values have the same meaning as the
memory locations \fhexr{0300}{030B} in the Atari. The maximum allowed
value for \fpara{length} is \fhex{0100}.

\begin{fcode}
SIO device, unit, command, direction, timeout, length, daux
\end{fcode} 

The \fcmd{SIO} command retains the parameters of the previous
execution, so they can be omitted in the next \fcmd{SIO} command. If all
parameters are omitted, the last \fcmd{SIO} is repeated. At the beginning, the
parameters are set to \fq{Get Status} for \fval{D1:}, so nothing wrong can
happen in case \fcmd{SIO} is entered without parameters.

In any case the
\fcmd{SIO} should always be used with caution and the parameters should be
verfied thoroughly, because otherwise formatting the wrong disk may happen
quickly.

\subsection{Display interrupt and VCOUNT}
The freezer can freeze the running program when an interrupt occurs.
To this end the freezer reroutes the interrupt vector and thereby forces the CPU
to enter the freezer. The \fcmd{V} command displays the type of interrupt (IRQ
or NMI) that caused activation of the freezer, together with the original value
of the corresponding interrupt vector. Additionally the value of \freg{VCOUNT}
at the moment of the freezer activation as well as the calculated value of
\freg{VCOUNT} at the moment the interrupt occurred are displayed. This command
was mainly introduced to test the freezer software and may be of little interest
for most users. But here's a short description of the significance and origin of
the values.

When the freezer is activated, it maps the ROM with the freezer software into
the address space of the CPU and \fq{reroutes} the high byte of the interrupt
vector to the ROM's address, so the freezer software is started.
Because the low byte is not changed, there's a full page of \fq{NOP}
instructions at the start of the freezer software. Therefore it may be the case
that the Atari has to execute through several \fq{NOP} instructions to get to
the part that will read the value of \freg{VCOUNT}.

Using the low-byte of the interrupt vector, the software tries to calculate the
amount of time it took to get through the \fq{NOP} instructions and adjusts the
\freg{VCOUNT} value accordingly. This way the program can be resumed at exactly
the same position where it was interrupted. This calculation is not 100\%
correct and may vary according to the graphics mode. In the very rare case that
the value is not correct and causes problems, the command 
\fcmd{V{\textless}value} can be used to set the \freg{VCOUNT} value manually.

\subsection{Activate direct I/O mode}
Changes to hardware registers only become effective after resuming, because in
the \linebreak \frz only the frozen system state is manipulated, as opposed to
normal debuggers which manipulate the real system sate.
For experts the \frz offers the possibility to activate the \fq{direct I/O mode}
and manipulate the real system state also in the debugger.

\begin{fcmdlist}
\item[DIO] Display current mode: \fval{0} = deactivated
(default), \fval{1} = activated
\item[DIO{\textless}0] Deactivate direct I/O mode
\item[DIO{\textless}1] Activate direct I/O mode
\end{fcmdlist}

\textbf{Caution:} When direct I/O mode is active, the freezer is
\fq{bypassed} and it becomes likely to make mistakes.  The freezer will not notice any changes
to the hardware registers in direct I/O mode. Therefore the same changes should
also be performed again with direct I/O mode switched off. Otherwise the freezer
will overwrite the changes when resuming.

\subsection{Display version number}
The command \fcmd{VER} prints the version number and the date of the freezer
software in the format \fmsg{3.10 2012-11-09}.

\subsection{Display OS vectors}
The \fcmd{VEC} command displays the most important OS vectors of page 0 and 2  as
an overview. The output contains the symbolic name of each vector and its value.
\begin{fcode}
DOSINI 000C: 0000   CASINI 0002: FFFF
DOSVEC 000A: F223
VPRCED 0202: C0CD   VINTER 0204: C0CD
...
\end{fcode}
\subsection{Display handler address table}
The \fcmd{HAT} command lists the entries of the handler table
(\fhexr{031A}{033A}). It displays the address of the entry, then the 3 bytes of
the entry followed by the device letter and the address of the handler table
(resp. \fq{\fmsg{-{}-{}~0000}} if the entry is empty).
\begin{fcode}
031A  50 30 E4  P: E430
031D  43 40 E4  C: E440
0320  45 00 E4  E: E400
\end{fcode}

\subsection{Display memory usage map}
The command \fcmd{M} displays a map of the memory which indicates used and unused
memory pages. If a page contains only zeros, a period \fq{\fmsg{.}} is
displayed. If it contains at least one non-zero byte, a star \fq{\fmsg{*}} is
displayed. This is useful for example when searching for an empty memory area
for the boot loader (see section \ref{sec:bootloader}).

\subsection{Activate printer output}
Printer output is controlled with the command \fcmd{PR{\textless}value}.
Values from \fval{1} to \fval{8} activate the output to printer \fcmd{P1:}
(default printer) to \fcmd{P8:}. The value \fval{0} deactivates printer output.
Via \fcmd{PR}, the current state of the printer output is displayed on the
screen. If printer output if active, everything that is displayed on the
debugger screen is also printed on the specified printer in parallel. This is a
handy way of logging debugger sessions.

Especially for longer debugger sessions it can be very useful to insert comments
into the logs. This can be achieved easily with the \fq{\fcmd{;}} command of the
debugger. Everything that follows the semicolon is printed in the debugger
window (and hence, if printer output is activated, also on the printer). This
way you no longer need to grab pen and paper frequently and the logs of the
debugging session will be understandable even months later.

\subsection{Control AtariSIO remotely}
Using the \fcmd{a} command, remote control commands can be sent to AtariSIO.
This enables a complete remote control of AtariSIO without loading and
additional program. More information can be found in the AtariSIO manual.
