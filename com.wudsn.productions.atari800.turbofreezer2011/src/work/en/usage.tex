\chapter{Usage}

Did you ever feel annoyed by a game that is really great but has no pause
function that can be activated at any time? Or by games of the frustrating kind
where you have to start all over again at the beginning after you lost all lives and
then have to play forever to get to the point that you had already reached
before? Or those games where the action only starts in the higher levels but
where you have to waste huge amounts of time reaching these levels? In this case
the freezer is exactly the right thing for you: a program can be frozen at any
time in any situation and can be saved to any mass storage. From there it can be
loaded at a later point in time and can be resumed at exactly the point where it
was frozen. And this can be repeated as often as required, so it is no problem to
spend a hundred lives to master a challenge though you only have one or two
lives left.

In order to really be fun, a freezer must be available at any time, must have
resident software (so it can work without cumbersome loading from disk or
cassette) and must work fully automatic and within seconds. Therefore no cost
cutting measures, like for example partly reconstruction of the hardware
registers by user input or loading the freezer software from disk, were applied.
Instead of applying such dubious cost cutting measures, as they can sometimes be
seen in the freezers for other computer systems, the \frz represents the best
possible solution that can be achieved with today's technology. And the
invested effort was not in vain, as the result proves.

If you know the cumbersome primitive freezers of other computers you will simply
be excited by the easy and instant usage of the \frz. For example freezing a
game, saving it to the RAM extension and resuming  the game three seconds later
takes no more than three keystrokes. The same is true for loading and resuming
the game from the RAM extension later on.

But the freezer can do much more. It is possible to perform any conversion
between cassette and disk. Users of cassette tape drives who recently bought a
disk drive can take their beloved software collection with them on disk. Users
of disk drives can now use programs that are only available on cassette without
tedious loading or can save money by buying the cheaper cassette
versions.

And there is one more important point. The \frz is a freezer which has a DOS and
debugger built in. They are always available and can be used without damaging
the frozen program. Full disks or other fatal events are now no longer a problem
anymore -- even not when using application programs without DOS functions. The
same is true for the evil bugs which can never be located without insight into
the hardware registers and the unmodified system state.

\section{Starting the freezer main menu}

After pressing the \fsw{Freeze} button the program is frozen when the next
interrupt occurs and the freezer takes over control of the Atari.From the
freezer main menu all other functions can be reached. Resuming the frozen
program is possible by pressing \fkey{SPACE}. The program will continue to run
from exactly the same position where it was frozen. The freezer main menu offers
the following options:

\begin{flist}
\item[\fkey{SPACE}]
  Resume frozen programm
\item[\fkey{RESET}]
  Perform cold start and deactivate freezer
\item[\fkey{S}]
  Save frozen program to disk or cassette
\item[\fkey{F}]
  Save frozen program to freezer RAM
\item[\fkey{R}]
  Save frozen program to RAM extension 
\item[\fkey{E}]
  Load and start frozen program from disk or cassette
\item[\fkey{C}]
  Load and start frozen program from freezer RAM
\item[\fkey{X}]
  Load and start frozen program from RAM extension
\item[\fkey{W}]
  Swap frozen program with program in freezer RAM and start
\item[\fkey{A}]
 Swap frozen program with program in RAM extension and start
\item[\fkey{Z}]
  Clear RAM under the OS ROM
\item[\fkey{D}]
  Start debugger
\item[\fkey{K}]
  Start menu of the cartridge emulation
\item[\fkeys{K}]
  Start menu of the cartridge emulation without clearing the RAM (warm start)
\item[\fkeys{SPACE}, \fkey{E}, \fkey{C}, \fkey{X}, \fkey{W}, \fkey{A}]
  Like above, but clear player/missile collision registers before resuming
\item[\fkeyc{E}, \fkey{C}, \fkey{X}, \fkey{W}, \fkey{A}]
  Like above, but frozen program is only loaded and not started
\end{flist}

\section{Freezing and resuming a program}

In principle, any program can be frozen at any point even during a disk or
cassette operation. Yet this is not recommended as this operation will remain
incomplete. After resuming the program the operation system has the opportunity
to retry the disk operation but this is not safe. Retrying an operation on the
cassette is even impossible for the operation system since the cassette tape
drive is simply too \fq{stupid}. Therefore it is better to freeze programs only
when there is no operation with peripherals in progress.

In very rare cases nothing happens when the \fsw{Freeze} button is pressed and the
program continues unaffectedly. This is the case when the program does not use
interrupts at all. It's only possible if the program is very simple like for
example converted Apple programs which do not really use the capabilities of the
Atari. To also freeze such programs you can install a \fkey{SYSTEM RESET} key in
the Atari as described in section \ref{sec:systemreset}. The interrupt triggered
by this key can by no means be suppressed, so there is no counter measure
against the \frz anymore.

When a frozen program is resumed with \fkeys{SPACE}, the player/missile collision
registers are cleared before control is handed over to the frozen program.
Usually this function is not required, but in some games this prevents losing a
life immediately after resuming. When loading frozen programs the same function
is also available. Simply press \fkeys{E} / \fkey{C} / \fkey{X} / \fkey{W} /
\fkey{A} to apply it.

When a frozen program is loaded from disk, cassette, freezer RAM or RAM
extension it is resumed by default. When loading a frozen program with \fkeyc{E}
/ \fkey{C} / \fkey{X} / \fkey{W} / \fkey{A} it will not be resumed
automatically. Instead you stay in the freezer and can for example change memory
locations before resuming. This comes very handy if you search for the memory location
that contains the number of lives, the amount of energy and so on. To this end
you can freeze the program and save it immediately. Then you change the memory
location and resume the program. If you used the wrong memory location, simply
activate the freezer and load the frozen program with \fkeyc{E}/\fkey{C}/\fkey{X}
and try the next memory location. This way you have the same starting point and
you don't have to worry about undoing the previous changes.

\section{Saving a frozen program}

The functions \fkey{S}, \fkey{F}, \fkey{R} can be used to save frozen programs
to external mass storage like cassette or disk (\fkey{S} key), to the RAM
extension (\fkey{R} key) or to the freezer RAM (\fkey{F} key). When using
\fkey{S} a submenu will prompt if the frozen program shall be saved to
cassette, as a boot disk or as a single file. Insert a cassette resp. disk
before confirming your choice. When saving as a single file to a disk an
additional menu will prompt for the file name.\newline

\textbf{Caution:} When you save the frozen program as a boot disk, the disk that
is present in \fq{D1:} will be overwritten directly. All data possibly present
on the disk will be lost in this case.\newline

If you simply press \fkey{RETURN} without entering a file name, the file name
\fval{CORE} for \fq{core dump} will be used. As a special feature wild cards can
be used to overwrite an existing file. If you omit the \fval{D:}, \fval{D1:},
\fval{D2:} etc. at the begin of the file name, \fval{D1:} will be used
automatically. The freezer supports up to 8 disk drives \fval{D1:} to \fval{D8:}.

In order to achieve the maximum possible speed the freezer menu is
sometimes disabled during I/O operations. By accepting that the screen may
\fq{jump} when it's switched back on again, saving to the RAM extension only
takes half of the time, what legitimates this dirty method.

\subsection{Adapting the boot disk loader}
\label{sec:bootloader}
When a frozen program is saved as boot disk, the freezer also writes a loader to
the disk. This inherently will not work with every program because the loader
itself requires about 2k of memory. If the frozen program requires already the
complete memory of the Atari, it will not work together with the loader.

By default the loader uses the area between \fhex{C000} and \fhex{C6FF}. This
normally guarantees correct operation with all programs that only require 48k
of memory. The start address of the loader can be adapted easily in case a
program still runs into problems.

The first byte of the boot sector contains the start page of the loader (by
default \fhex{C0}). Adapting this byte is very easy with the \frz.
Start the debugger, load the first sector, change the byte to the desired value
and write the sector back to the disk. Please note that only values from
\fhex{05} to \fhex{C9} and from \fhex{D8} to \fhex{F9} are allowed. If you use
other values, you will obtain a \fmsg{BOOT ERROR} when booting the disk.
The following sequence of commands can be entered in the debugger to adapt the
start page of the loader for example to \fhex{F0}:

\begin{fcode}
SR 1
C D700<F0
SW 1
\end{fcode}

\subsection{Freezing a program that loads parts later }
Freezing programs that load parts later is of course always possible. The only
thing to remember is that the original disk must be inserted into the disk drive
after resuming the program and before any further operation is performed in the
running program. This of course wears out the disk over time. Therefore it is
desirable to have a backup disk at hand so the original disk can be locked away
in a safe location. But even with floppy speeders that include a copy function
not all new programs can be copied by far.

The \frz does not remove the copy protection from the original disk, it only
saves the current program state. In order to resume from this state later as
often as you like, the original copy protected disk must be inserted into the
disk drive again, before the program is resumed. If you use an additional disk to
save the frozen game state, the original disk should be equipped with a write
protection. This prevents accidentally overwriting the original disk.

If the game loads additional parts from multiple disks without copy protection
(\eg an adventure), it is advisable to copy these disks and use the copies
instead to not wear the original disks out.
\clearpage

\section{Swapping the frozen and a saved program}
The swap function in the freezer main menu can be used to swap the current
frozen program with the program that was saved to the freezer RAM (\fkey{W}
key) or the RAM extension (\fkey{A} key) before. This way you can jump quickly
between two different programs. You should be sure to really have a frozen
program in the freezer RAM resp. in the RAM extension, otherwise the Atari will
crash when resuming.

The modifier keys \fkeys{} for clearing the player/missile collision registers
and \fkeyc{} for loading without starting automatically can be used with the
swap function just like with the normal functions for resuming.

\section{Clearing the RAM under the OS ROM}
Because there are programs which require and also use 64k RAM, the freezer must
also take the RAM under the OS ROM into consideration. After switching the
computer on, this area is filled with useless random data. Because this data is
ineligible to compression, a certain inefficiency and waste of space are the
consequence when saving 48k programs to an external medium. In order to avoid this,
the RAM from \fhexr{C000}{FFFF} can be cleared with the function\fkey{Z}, in
case the program only uses 48k RAM. The result is a reduction of the file size
of about 25\% to 50\% and consequently shorter loading times.

The function can be used before booting the program or before saving the
program. The first alternative is recommended  if you are not completely sure that
the program really does not store anything in this RAM area.


\section{Starting debugger and DOS}

The function \fkey{D} starts the built-in debugger with DOS functions. The
debugger uses a command line and the complete screen, so controlling it via a
menu would not be sufficient. Yet no RAM in the Atari is used or modified by
this function. The states of the frozen hardware registers of course also remain
intact.  The complete description of the debugger and the DOS function can be
found in chapter \ref{chap:debugger}.

\section{Starting the menu of the cartridge emulation}

The function \fkey{K} starts the menu of the cartridge emulation. This function
is only available if the \fsw{CartEmu} switch or the \fsw{FlashWrite} switch (or
both) are on, \ie are turned right. Otherwise the cartridge emulation is
completely deactivated and consequently the menu of the cartridge cannot be
started. The complete description of the cartridge emulation can be found in
chapter \ref{chap:cartemu}. The freezer performs the following steps when the
menu of the cartridge emulation is started:
\clearpage

\begin{itemize*}
\item Leave the freezer main menu
\item Activate the menu of the cartridge emulation
\item Disable IRQs and NMIs
\item Enable the OS ROM
\item Set the return address to \fhex{E477} (cold start)
\item Resume as usual (corresponds to pressing \fkey{SPACE})
\end{itemize*}

With the function \fkeys{K} a warm start is performed instead of a cold start
(\ie the return address set to \fhex{E474}). This can be useful to keep the
content of the internal RAM of the Atari when starting a module.
