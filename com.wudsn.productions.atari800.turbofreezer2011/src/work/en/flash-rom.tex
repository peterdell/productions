\chapter{Programming the flash ROM}

\section{Basics}
The software of the \frzs is not, as this is normally the case, stored in an
EPROM but in a flash ROM. The big advantage of the flash ROM is that it can be
re- programmed directly via the Atari. This means software updates can be
installed at any time and the unused parts of the flash ROM can be filled with
module data without the need for a special programming device. The flash ROM in
the \frz can be re-programmed up to 1.000.000 times, which should be enough for
most experiments.

The flash ROM and the freezer RAM of the \frz are normally protected against
overwriting. The \fsw{FlashWrite} must be moved to the right position
(\fval{ON}) to enable write access to both memories. If the switch is in the left position
(\fval{OFF}), the flash ROM and the freezer RAM are write protected. It is very
unlikely that any software overwrites the flash ROM accidentally. To program
the flash ROM a special write sequence must be sent to the flash ROM, otherwise
write access is simply ignored. Therefore is not really dangerous to have the
\fsw{FlashWrite} switch in the right position (\fval{ON}) permanently.

When using \fq{real} bank switching cartridges in the cartridge port both the\linebreak
\fsw{CartEmu} as well as the \fsw{FlashWrite} switch should be turned \fval{OFF}.
This prevents a conflict between the cartridge emulation and the bank switching
cartridges.\newline

\textbf{Caution:} It is important that the \fsw{OldOS} switch is turned
\fval{OFF} during flashing in order to ensure that the flash ROM is exclusively
accessible by the flashing program. If the Oldrunner is active at the same time
conflicts can occur -- \eg when an interrupt is raised -- and the Atari can crash.

\section{Banks and blocks in the flash ROM}
It is helpful to know some details about the inner structure of the flash
ROM in order to use it best. The used flash ROM is partitioned into blocks of
64k each (\ie eight 8k banks). These blocks can be re-programed completely
independent of each other. This means that \eg changes to the data of the
cartridge emulation to not impact the freezer software and vice versa.

Unfortunately the partitioning into 64k blocks also has a small disadvantage. It
is not possible to re-program only a part of a 64k block. Still free, not yet
programmed parts can be programmed later on. You can for example first program
an 8k module into bank 0 and program later another 8k module into bank 1 of the
same block. But is it recommended to merge several \eg 8k modules into a large
file and program them then in one go. The restriction to 64k blocks does of
course not apply to the freezer RAM. There every 8k bank can be programmed
separately.

If the start bank is not at a 64k block boundary, then the flash program asks if
the complete 64k block shall first be erased or not. If the 64k block is not
erased, unused (\eg previously erased) parts of a 64k block can be filled with
data. If you accidentally try to overwrite already used parts, an error message
appears and you have to re-program the complete 64k block.

\section{Starting the flash program}
To program the flash ROM resp. the freezer RAM the \fsw{FlashWrite} switch must
be turned right (\fval{ON}). The program \fcmd{FLASH.COM} must be loaded
from the system disk. If the switch is turned \fval{OFF}, then the program will not
find the flash ROM and will output an error message.
In this case you can turn the switch right and confirm the question
\fmsg{restart program?} with \fkey{Y}. The flash program outputs the following
information when it starts:

\begin{itemize*}
\item Type of the flash ROM
\item Version number and date of freezer software in the flash ROM
\end{itemize*}

In case no freezer software is present in the flash ROM, the output reads
\fval{n/a} instead of the version number. The version number can also be
displayed in the debugger via the \fcmd{VER} command.

\section{Using the flash program}
The flash program offers the following operations:

\begin{fcmdlist}
\item[Program CartEmu flash] Program the flash ROM area of the cartridge
emulation (banks \fdecr{0}{119}). The area for the freezer software (banks
\fdecr{120}{127}) remains protected in this case and cannot be overwritten
accidentally. If you only work with the cartridge emulation, then you should
always only use this operation.
\item[Program CartEmu+Freezer flash] Program the complete flash ROM (banks
\fdecr{0}{127}). With this you can update also the freezer software (banks
\fdecr{120}{127}).
\item[Write flash to file] Write the flash ROM content to a file.
\item[Program RAM] Program the freezer RAM (banks \fdecr{0}{47}).
\item[Write RAM to file] Write the freezer RAM content to a file.

\clearpage
\item[Erase flash] Erase the complete flash ROM.\newline

\textbf{Caution:} This will erase the complete flash ROM, including the freezer
software. In order to then use the \frz in a normal fashion you have to first
re-program the freezer software again starting at bank 120. Otherwise the Atari
will crash if the \fsw{Freeze} button is pressed or if the cartridge emulation
or the Oldrunner is activated.
\item[Start CartEmu] Jump to the menu of the cartridge emulation. With this you
can test the module you have just programmed.
\end{fcmdlist}

\section{Programming the flash ROM and freezer RAM}
The following steps are performed when programing the flash ROM resp. the
freezer RAM. At first the flash program asks the user for the first bank number
to be programmed. Valid values are \fdecr{0}{127} for the flash ROM and
\fdecr{0}{47} for the freezer RAM. Then the flash program asks for the number
of 8k banks to be programmed. If you confirm the default value \fval{0}, the
complete file will be programmed automatically. 

Now the name of the file which contains the data to be programmed must be
entered. The file must only contain the data to be programmed, \ie it must not
contain a COM header or something similar and the file size must be a multiple
of 8k. The single 8k banks of the flash ROM (or the freezer RAM) are
now programmed sequentially. When programing the flash ROM, every 64k block will
be erased when a 64k block boundary is reached. Programming will also stop if
the end of the flash ROM or freezer RAM area is reached before the end of the
file is reached.

During keyboard input and during programming, the current operation can be
aborted with the \fkey{BREAK} key at any time.

\section{Updating the freezer software}
If the freezer software in the flash ROM was overwritten or erased
accidentally, or if the software shall be replaced by a newer version, then
the file \fval{FREEZER.ROM} from the system disk must be programmed to the banks
\fdecr{120}{127} of the flash ROM. To this end the following steps are required:

\begin{enumerate*}
\item Turn the \fsw{FlashWrite} switch right (\fval{ON}).
\item Load the flash program \fcmd{FLASH.COM} from the system disk.
\item Choose the 2nd menu item (\fmsg{Program CartEmu+Freezer
flash}).
\item Enter start bank \fval{120}.
\item Confirm the default value \fval{0} for the number of banks with \fkey{RETURN}.
\item Enter \fval{FREEZER.ROM} as file name.
\end{enumerate*}
