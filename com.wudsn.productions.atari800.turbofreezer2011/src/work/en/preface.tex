\chapter*{Preface}
\markboth{}{Preface}

\section*{Preface by the designer of the new freezer (2012)} \frz 2005
was a big success. It was very well received in the Atari community and we got
lots of positive feedback and ideas for new features. The decision to use a
flash ROM instead of an EPROM proved being particularly helpful. It enabled
users to enjoy software updates and new features as the flash ROM could be
updated from the Atari.

In particular the debugger has been extended considerably since the first
release 2005. An online help, commands for moving and comparing memory blocks, a
memory usage map and an OS vector overview, the possibility to log the screen
output of debugging sessions to printer in parallel and many additional things have
been added. Special thanks go to Erhard P�tz and Carsten Strotmann for many
good ideas.

The initial run sold out fairly quickly and requests for more kept coming. In
2010 things became more concrete. Together with Wolfram Fischer, who also
participated in the first run, I started planning a new production run.
Meanwhile the Lattice M4A5 CPLD had become relatively hard to get and was
clearly more expensive than comparable chips. We also wanted to use a slightly
\fq{larger} chip in order to implement some new ideas and features. So we
decided to use a Xilinx XC95144XL as the basis for the new freezer. Having about
the same price it offered more than twice the power (I/O pins and space for
logic) than the Lattice CPLD. This way we had the possibility of implementing
long-awaited features -- for example the RAM extension which had been there in
the very first freezer from 1987.

By the end of 2010 Wolfram had created the first prototype of the new hardware.
This was the basis for starting the development of the new logic and software at
the begin of 2011. I realized quickly that the best way would be to develop the
logic from scratch and to separate the functional blocks (freezer, cartridge
emulation, etc.). This simplified extensions in each functional block.

The logic of the cartridge emulation was completely  redone; the freezer logic
has intentionally been kept compatible with old \frz 2005 logic. I
wanted to keep the changes to the freezer software at a minimum, so it can also
be used with the \fq{old} freezer. This way the users of the old freezer may
also benefit from further updates.

In summer 2011, after several new hardware revisions and many changes to the
software and the logic, the new freezer was completed! Unfortunately the start
of the mass production, which had been planned for end of 2011, had to be
postponed by almost one year. Wolfram had to move to a new residence and was
busy at his job -- hence we both had almost no time left for our hobby. Thanks
to the support from the ABBUC the production finally started in autumn 2012!
Many thanks to Wolfgang Burger for the support. Also thanks to all who ordered
the new freezer in advance and waited so patiently.

\medskip Salzburg, November 2012

\medskip Matthias Reichl

\medskip P.S. (March 2013): I'd like to add a big \fq{thank you} to Peter Dell
who took over the job of translating the German manual into English.
At the end this was a whole lot more of work than expected, but Peter
was really brave and did an awesome job -- thanks Peter!

\section*{Preface by the designer of the new freezer (2005)} The project to
design a new \frz started in early 2004, when Bernhard Engl gave
permission to the ABBUC to remake his products (The TURBO 1050 and the \frz).
To me, the \frz has always been the most interesting piece of
hardware extension that has ever been available for the Atari and before
starting the project, I had already studied and analyzed how it worked.
That's why it was pretty obvious that I would engage in rebuilding the \frz{}.
There were more participants in the team: Bernhard Pahl, Florian
Dingler, Torsten Schall, Guus Assmann, Frank Schr�der and towards the end also
Wolfram Fischer.

So I would like to express big thanks to all of you. The project wouldn't have
come this far without you!! And some special thanks to Guus, for the many ideas
for extensions, designing the PCBs and producing the prototypes. Thanks as well
to Bernhard Engl for his tips and the detailed information for the original
\frz. All other team members, thanks for the support and the testing!

My original plan was to remake the freezer, using modern parts. At this point, I
never imagined that the result would be such extensive and powerful
improvements! During the project, both hardware and software have been gradually
extended and improved up to a point where only the basic (and genius) principle
idea of the freezer was left. Everything else had been changed.

Some of the extensions came by as pure coincidence. For example the cartridge
emulation: As 16K EPROMs were much more expensive, Guus suggested using a flash
ROM. To enable programming this flash ROM from the Atari, it was needed to map
the chip into the memory-map of the Atari.
This formed the basis for the cartridge emulation :-)

Likewise with the RAM: The original 2K RAM was hard to get and also quite
expensive. So the first step was to upgrade to a 32K chip and also use 128K
flash ROM. The next step was to go to 128K RAM and this made it possible to put
a memory-snapshot into this RAM. And the flash ROM size was increased to 512K.
To enable this, the logic had to be improved. And a bigger logic chip gave space
for the emulation of the 16K OSS modules and the SpartaDOS~X module.

With the current state of the logic chip's internals, there's no space left and
we're all content to have gotten the maximum out of the parts that were used.
For the software, there are some more ideas for extensions, but that's all I'll
say for now.

During the development process, I've constantly kept in mind that I wanted to
create an open basis for further developments. For this reason, all design
information is freely available and there's even a JTAG interface on board
so the logic can be reprogrammed. This will help anyone who wants to expand the
freezer or even make something completely new with it.

All relevant information and updates for the \frz XL/XE 2005 are
available on the internet at \url{http://turbofreezer.horus.com/}. There's also
a manual on how to alter the logic of the \frz. For technical
questions, write an e-mail to
\href{mailto:hias@horus.com}{\nolinkurl{hias@horus.com}}.

The many features of the \frz have not only convinced the team
members, but also many members of the ABBUC. In their Hardware Contest 2005, it
was awarded a very convincing first prize.

I wish you a lot of fun using the \frz and stay loyal to the Atari
8-bit computers for many years to come.

\medskip Salzburg, December 2005

\medskip Matthias Reichl.

\section*{Preface by the developer of the original freezer (2005)} As a
designer, the best recognition of achievement one can get, is that a once
successful product is continued and enhanced by a competent successor. However,
if this happens after 16 years and for a home-computer product, it's downright
sensational and means that the product has attained cult status. While using
original concept of the 1980's, Matthias Reichl has made a real masterpiece of
the \frz XL/XE 2005. The application of modern micro-chips has led to
a great extension of the functionality. He even managed to remove some minor
bugs that were known, but couldn't be prevented using the electronics of that
time, without making the product economically impossible because of the higher
chip-count.

The improvements and enhancement, especially the ingenious cartridge emulation
that was not present in the original, make the \frz XL/XE not only
the best and most perfect freezer for the 8-bit Atari that ever existed, but
also is an all-rounder talent, having the potential to become a legendary cult
product. At this point in time, it's a high-light of a technical development
that started more than 20 years ago. In my notebook, which has turned yellow
now, the first sketches of the concept date back to the autumn of 1984. Finding
this out even surprised myself.

Experiencing a vastly improved, enhanced and rejuvenated new series of a
computer extension product, is not only a miracle, but also a special joy and
honor to me, so I wish all proud owners of the new \frz XL/XE a lot of
pleasure using it.

\medskip Zurich, November 2005

\medskip Bernhard Engl

\section*{Preface by the developer of the original freezer (1987)}
After one year development, hundreds of used-up integrated circuits, 5300 lines
of assembler code and five-digit costs, it's there at last:
The \frz XL. It's not only the first real freezer for the Atari, it is
also the only extension a freak needs to get the optimum and maximum usage out
of his Atari. This is because apart from freezer, the board also offers a socket
for an Oldrunner and for 256k of RAM so an 800 XL can be extended to up to 320k.

By using three ASICs containing the equivalent of over 40 TTL circuits and by
avoiding a costly multi-layer PCB, all this could be implemented at an unbeatable
price. In addition, thanks to the very powerful memory management that is anyway
needed for the freezer, the computer doesn't need to be opened.

The careful same tuning of all components can be found in the built-in software
which consists of the freezer, a mini-DOS and a debugger. Combined with the
hardware, this power-tool gives an unprecedented possibility of control over
programs to the user. That's why anyone who's seen the \frz XL in
action, wanted to have one as well. The thrill of having total control over the
computer, even though the programmer did not want this, is well worth the price.
And then there's also the money save from buying separate extensions that would be
needed to get the same functions that are present in the \frz XL.

More about this in the description of the individual functions! I wish every
proud owner lots of fun with the most fascinating product that was ever
available for the Atari.

\medskip Munich, May 1987

\medskip Bernhard Engl
