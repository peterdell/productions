\chapter{Cartridge emulation}
\label{chap:cartemu}
The cartridge emulation is another very powerful feature of the \frz.
It is capable of emulating standard 8K and 16K modules as well as bank switching
modules according to the AtariMax/MegaMax and OSS standards. Additionally the
new\linebreak
SpartaDOS~X (Ultimate1MB bank switching, up to 512k) can be emulated.
Also combined usage of SpartaDOS~X together with an OSS module is allowed.

The cartridge emulation can use all unused memory of the flash ROM (960k) and
the freezer RAM (384K). This means the \frz gives you instant access to up to
168 different modules.
The cartridge emulation is not only interesting for those who use many different
cartridges and who became tired of changing the cartridges all the time (which
heavily strains the cartridge port), but also for those who develop cartridge
based software.
Because the module data can not only be stored in the flash ROM but also in the
freezer RAM, changes to the data can be performed and tested within seconds.
Since the freezer RAM is battery backed its content is not lost after
switching the Atari off. This means the freezer RAM can be used for storing
permanent data, just like the flash ROM.

\section{Basics}
Leveraging the full potential of the cartridge emulation requires knowledge
about the following details of the internal operation.

\subsection{Bank numbers}

The flash ROM is internally divided into \fq{banks} of 8K.
With 1MB flash ROM, there are 128 8k banks with the bank numbers
\fdecr{0}{127} available. The topmost 64k of the flash ROM are used by the
freezer software and cannot be used for cartridge emulation. This leaves the banks with the numbers
\fdecr{0}{119} for cartridge emulation.

In the 512k freezer RAM the topmost 128k are reserved for the freezer software,
leaving 384k in the banks\fdecr{0}{47} for the user. Snapshots are stored in the
banks \fdecr{48}{55} of the freezer RAM. If you don't use snapshots, you can
use these banks also for cartridge emulation.

16k modules use 2 consecutive 8k banks. Their data must be aligned to a 16k
boundary, i.e. an even bank number. If a 16k module is for example stored in the
banks 2 and 3, then bank 2 will be visible at \fhexr{8000}{9FFF} and bank
3 will be visible at \fhexr{A000}{BFFF}.

512k modules, SpartaDOS~X and modules with \frz 2005 bank switching (\fval{8k
old}) must be aligned to a 512k boundary. That means they can only start at bank
0 or 64 in the flash ROM or at bank 0 in the freezer RAM.

The SpartaDOS~X emulation uses the same bank switching scheme as the\linebreak
\fq{Ultimate 1MB} extension. The corresponding image has a maximum size of
512k. Currently the SpartaDOS~X~4.45 image for the Ultimate 1MB extension has
256k and can therefore be used without problems at bank 0 or 64 in the flash ROM
or at bank 0 in the freezer RAM.

\subsection{Module types}
The module type defines at which address in the Atari the data from the
cartridge emulation shall be visible. Also additional bank switching registers
(e.g. for OSS modules) are activated the area \fhexr{D500}{D5FF}, depending on
the module type.

The following module types are supported:

\begin{fcmdlist}
\item[8k]
8k standard module, \fhexr{A000}{BFFF}.\newline
The software can access the whole memory, i.e. the flash ROM and the freezer
RAM, via the bank switching register of the cartridge emulation.
\item[8k+RAM]
8k module, \fhexr{A000}{BFFF}\newline
with optional 8k RAM bank at \fhexr{9000}{BFFF}. Both areas can be selected independently 
via separate bank registers.
\item[8k old]
512k bankswitching module, \fhexr{A000}{BFFF}.\newline
\frz{} 2005 compatible bank switching.\newline
Bank 0 or 64 must be used as start bank.
\item[8k AtariMax]
1MB (8MBit) bank switching module, \fhexr{A000}{BFFF}.\newline
Bank switching according to the AtariMax/MegaMax standard. \newline

\textbf{Caution!} The cartridge emulation offers 64k less than the
AtariMax 8Mbit module. Therefore modules which require the complete 1MB of
memory will not work, like for example Space Harrier. When creating own modules
with the MaxFlash software, the maximum size of 960k must not be exceeded. In
addition the bank 0 must be used as start bank.

\item[16k]
16k standard module, \fhexr{8000}{BFFF}.\newline
Bank switching is possible like in the 8k mode.
\item[OSS]
16k OSS bank switching module, \fhexr{A000}{BFFF}.\newline
For example MAC/65 or ACTION!.


\end{fcmdlist}

\section{Activation of the cartridge emulation}
To activate the cartridge emulation, move the \fsw{CartEmu} switch to the right
position (\fval{ON}) while the Atari is switched off. When switching the
Atari on, a small menu is displayed where the cartridge emulation can be
configured.

Additionally you can enter the cartridge emulation menu from the\linebreak
\frz
menu via the key \fkey{K} or \fkeys{K}. This works only if the cartridge
emulation is currently active, i.e. at least one of the switches \fsw{CartEmu}
or \fsw{FlashWrite} is in the right position (\fval{ON}). The cartridge emulation is
completely disabled if both switches are in the left position (\fval{OFF}).

\section{Cartridge emulation menu}
The current configuration is displayed in the middle of the screen.
The online help with the available options is displayed at the bottom of the
screen.

\begin{fcmdlist}
\item[MODE]
Selects the emulated module type. \newline
\fval{OFF} deactivates the cartridge emulation. \newline
\fval{PicoDos} starts the integrated MyPicoDos instead of a module.

\item[SRC]
Selects the flash ROM or freezer RAM as source for the module data.

\item[BANK]
Selects the flash ROM or freezer RAM start bank.

\item[SDX]
Selects the SpartaDOS~X emulation. You can choose between
\fval{OFF},\linebreak
\fval{ROM~Bank~0}, \fval{ROM~Bank~64} and \fval{RAM~Bank~0}.

\item[BOOT]
Selects if the module shall be started via a cold start (\fval{COLD}) or
a warm start (\fval{WARM}). This option should normally be set to 
\fval{COLD}, because otherwise the operating system variables and the module are
not correctly initialized. The option \fval{WARM} is intended for developers and only makes
sense if the cartridge emulation was activated from the \frz menu via
 \linebreak \fkeys{K}. This way the cartridge emulation can be reconfigured
while keeping the internal RAM of the ATARI unchanged. This can be useful during
development.
\end{fcmdlist}

Some frequently used configurations can be selected with a single key press.

\begin{flist}
\item[\fkey{D}]
Selects the default configuration: \fcmd{Mode}~\fval{8k}, \fcmd{Bank}~\fval{0},
\fcmd{Source}~\fval{ROM}, \fcmd{SDX}~\fval{OFF}
and \fcmd{Boot}~\fval{Cold}.

\item[\fkey{P}]
Selects the \fcmd{Mode}~\fval{PicoDos}.

\item[\fkey{O}]
Deactivates the cartridge emulation: \fcmd{Mode}~\fval{OFF}, \fcmd{SDX}~\fval{OFF}
\end{flist}

Pressing the \fkey{RETURN} key activates the currently selected
configuration. Pressing \fkey{ESC} deactivates the cartridge emulation and
restarts the Atari.

\section{Command summary}
The debugger supports the following commands, which all begin with \fcmd{K}, to
control the cartridge emulation. The commands have the same effect as
manually
chaning the configuration registers starting at \fhex{D500}, but some
people are said to prefer mnemonics over manual entry of hexadecimal values.

\clearpage

\begin{fcmdlist}
\item[K]
Display cartridge emulation configuration
\item[K{\textless}0]
Deactivate cartridge emulation completely: \\
Sets module type and SpartaDOS~X emulation to \fval{OFF}, all banks to \fval{0}
and source to \fval{ROM}
\item[KM{\textless}type]
Sets module type of the cartridge emulation. Possible values:
\begin{fvallistn}
\item[0] off
\item[8] 8k module
\item[8R] 8k module with optional 8k RAM bank at \fhex{8000}
\item[8O] 8k module with \frz 2005 banks witching
\item[A] 8k AtariMax/MegaMax module
\item[16] 16k module
\item[O] 16k OSS module 
\end{fvallistn}
\item[KB{\textless}bank]
Select 8k main bank, (\fhexr{00}{7F}).
\item[KBE{\textless}mode]
Switch 8k main bank on/off:
\begin{fvallistn}
\item[0] off
\item[1] on
\end{fvallistn}
\item[KR{\textless}bank]
Select optional 8k RAM bank (\fhexr{00}{3D})
\item[KRE{\textless}mode]
Switch optional 8k RAM bank on/off:
\begin{fvallistn}
\item[0] off
\item[1] on
\end{fvallistn}
\item[KX{\textless}mode]
Select SpartaDOS~X emulation mode:
\begin{fvallistn}
\item[0] off
\item[1] on and in flash ROM at bank 0
\item[2] on and in flash ROM at bank 64
\item[3] on and in freezer RAM at bank 0
\end{fvallistn}
\item[KXB{\textless}bank]
Select SpartaDOS~X emulation bank (\fhexr{00}{3F}).
Only possible, if emulation was activated via \fcmd{KX}
\item[KXM{\textless}mode]
Select SpartaDOS~X mode. Only possible, if emulation was activated via
\fcmd{KX}
\begin{fvallistn}
\item[0] SpartaDOS~X off, additional module off
\item[1] SpartaDOS~X on, additional module off
\item[C] SpartaDOS~X off, additional module on
\end{fvallistn}
\item[KO{\textless}bank]
Select OSS bank. Only possible, if emulation was activated via
\fcmd{KM{\textless}O}
\begin{flistn}
\item[\fval{0}] OSS module off
\item[\fval{1}\dots\fval{3}] Select OSS bank \fval{1}\dots\fval{3}
\end{flistn}
\item[KW{\textless}mode]
Switch write access to cartridge emulation on/off:
\begin{fvallistn}
\item[0] off
\item[1] on
\end{fvallistn}

\clearpage

\item[KS{\textless}source]
Select source for the cartridge emulation:
\begin{fvallistn}
\item[0] flash ROM
\item[1] freezer RAM
\end{fvallistn}
\item[KN{\textless}mode]
Switch menu for cartridge emulation on/off:
\begin{fvallistn}
\item[0] cartridge emulation normal, menu deactivated
\item[1] cartridge emulation off, menu activated
\end{fvallistn}
\end{fcmdlist}

\section{Further information}

\subsection{Using \fq{real} cartridges}

When using a \fq{real} module in the cartridge port of the Atari 600~XL / 800~XL
or in the cartridge port of the Atari XE adapter board of the \frz, the
cartridge emulation should be deactivated completely. To this end the
\fsw{CartEmu} as well as the  \fsw{FlashWrite} switch must be moved to the left
position (\fval{OFF}). Otherwise the system may not work correctly, particularly
if the cartridge uses bank switching.


\subsection{TRIG3 -- \fhex{D013}}

Compared to the cartridge emulation of the \frz 2005, a small but important
detail in the cartridge emulation of the \frz 2011 has been improved.
If the cartridge emulation is active, the presence of a module is correctly
reported when reading the \freg{TRIG3} (\fhex{D013}) register. This means the
new cartridge emulation behaves 100\% like a \fq{real} module and patching
of module images, which was necessary for the \frz 2005, is no longer required.
