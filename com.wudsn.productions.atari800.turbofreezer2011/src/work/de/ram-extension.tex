\chapter{512k RAM-Erweiterung}

Die eingebaute RAM-Erweiterung des \frzs ist kompatibel zu den verbreiteten
Rambo/AtariMagazin Erweiterungen. Bit 4 von \freg{PORTB} aktiviert den Zugriff
auf die RAM-Erweiterung (0~=~ RAM-Erweiterung bei \fhexr{4000}{7FFF} einblenden,
1 ~=~ internen Atari Speicher verwenden). Die Bits 2,3,5,6 und 7 w�hlen eine
der 32 16k B�nke aus. Ein separater ANTIC Zugriff, wie beim Atari
130 XE und einigen anderen RAM-Erweiterungen vorhanden, wird nicht
unterst�tzt. Zum einen liegt das dazu ben�tigte HALT Signal nicht am
PBI an. Zum anderen gibt es nur ganz wenige Demos, welche diesen Modus
�berhaupt verwenden und die meisten davon sind mittlerweile auch f�r
RAM-Erweiterungen ohne separaten ANTIC Zugriff angepasst worden.

Beim Einsatz von Programmen, die eine RAM-Erweiterung mit separatem ANTIC
Zugriff erfordern, ist folgende Einschr�nkung zu beachten. Befindet sich im Atari eine
interne RAM-Erweiterung, die den separaten ANTIC Zugriff unterst�tzt, kann es
bei aktivierter 512k RAM-Erweiterung am Freezer zu Fehlfunktionen kommen. Ist
nur der ANTIC Zugriff auf die RAM-Erweiterung aktiviert (�ber Bit 5 von
\freg{PORTB}), nicht aber der CPU (bzw. kombinierter ANTIC/CPU) Zugriff (�ber
Bit 4), so greift der ANTIC auf die interne RAM-Erweiterung zu statt auf die
RAM-Erweiterung am Freezer. Die beste L�sung ist diese Situation erst gar nicht
auftreten zu lassen und eine der beiden RAM-Erweiterungen zu deaktivieren.
