\chapter{Debugger und DOS}
\label{chap:debugger}
Vom Freezer-Hauptmen� gelangt man mit der \fkey{D} Taste in den eingebauten
Debugger mit DOS Funktionen.

Die Eingabe von Kommandos erfolgt immer in der Kommandozeile unten am
Bildschirm. Innerhalb der Kommandozeile ist ein Editieren in der �blichen Weise
m�glich. Der Cursor kann die Kommandozeile aber nicht verlassen. Mit den Pfeiltasten
\fkey{CURSOR UP} und  \fkey{CURSOR DOWN} kann man die zuletzt eingegebenen
Kommandos wieder aufrufen. Der Freezer merkt sich hierbei die 4 letzten
Kommandos.

Der Debugger unterst�tzt auch die sonst leider viel zu selten verwendete
\fkey{HELP} Taste. Durch einen Druck darauf wird eine kurze Hilfe mit einer
�bersicht aller Kommandos angezeigt.
Mit den Tasten \fkey{1}, \fkey{2}\dots springt man zu den einzelnen Hilfeseiten,
mit \fkey{DEL} geht man eine Seite zur�ck, mit \fkey{Leertaste} oder \fkey{HELP}
eine Seite vor. Jede andere Taste beendet die Hilfe. Der Inhalt der
Kommandozeile bleibt erhalten, man kann die Hilfe aufruft. Das hei�t, man kann
sie auch w�hrend der Eingabe eines Kommandos aufrufen wenn man die genaue Syntax
nicht mehr im Kopf hat.
