\chapter{Oldrunner -- Atari OS Rev. B}
Der Atari 400/800 ist der einzige 8-Bit Homecomputer, der ein sauber
gegliedertes und durchdachtes Betriebssystem (OS) enth�lt. Es erm�glicht
jederzeit �nderungen, Erweiterungen und Verbesserungen, ohne dass vorhandene
Programme angepasst werden m�ssten. Daher konnte Atari bei der XL-Serie ein
neues, leistungsf�higeres OS verwenden. �rgerlicherweise existieren doch
Programme (Jahrgang 1980-1983), die auf der Atari XL/XE Serie nicht
funktionieren. Die Schuld liegt bei den Programmierern der betroffenen
Programme, die die offiziellen Programmierrichtlinien nicht eingehalten hatten.

Um die unvertr�glichen Programme trotzdem auch auf den Atari XL/XE laufen lassen
zu k�nnen, wurden sogenannte \fq{Translator} Disketten herausgebracht.
Diese schalten das OS-ROM ab und kopieren die alte OS-Version \fq{Rev. B} in das
das unter dem OS-ROM befindliche RAM. Diese Softwarel�sung funktioniert zwar in
den meisten F�llen, ist aber wegen des dauernden Ladens unbequem und
zeitraubend. Ideal ist nur eine Hardwarel�sung, bei der das alte OS auf einem
ROM abgelegt ist und damit jederzeit unver�nderlich zur Verf�gung steht. Nur
diese L�sung funktioniert mit allen unvertr�glichen Programmen. Leider
erforderten solche sogenannten \fq{Oldrunner} normalerweise Eingriffe in den
Atari.

Mit dem \frz ist es nun m�glich, einen Oldrunner ohne Eingriffe in den Atari zu
realisieren. Die Memory-Management-Logik macht dies m�glich. Mit dem Schalter am
\frz, der mit \fsw{OldOS} beschriftet ist, kann der Oldrunner ein- und
ausgeschaltet werden. Eingeschaltet ist er, wenn der Schalter in der rechten
Position  \fval{ON} steht. Es ist ratsam, diesen Schalter nicht bei
eingeschaltetem Atari zu bet�tigen. Ansonsten passt der Inhalt des RAMs nicht
zur neu ausgew�hlten OS-Version und das System st�rzt ab.

Aufgrund einiger Umst�nde ist es ratsam, nur dann den Oldrunner zu aktivieren,
wenn das Programm anders nicht l�uft. Beim aktivem Oldrunner gibt es kein
eingebautes BASIC, keine RAM-Erweiterung und keinen Warmstart.
Die \fkey{RESET} Taste beim Atari XL/XE bewirkt bei aktivem Oldrunner immer (!)
einen Kaltstart und der Inhalt des RAMs geht verloren. Die dem \fkey{RESET} beim
XL/XE entsprechende Taste beim Atari 400/800 hie� \fkey{SYSTEM RESET} und
bewirkte einen nicht-maskierbaren Interrupt (NMI). Sie kann zwar ohne gro�en
Aufwand nachger�stet werden (siehe Abschnitt \ref{sec:systemreset}), das
erfordert aber einen Eingriff in den Atari und ist damit Experten vorbehalten.
Abgesehen davon ist diese Taste ohnehin nicht unbedingt notwendig, da die
meisten unvertr�glichen Programme Spiele sind, die den \fq{SYSTEM RESET} in vielen
F�llen auf Kaltstart oder Absturz programmiert haben. Es handelt sich dabei um
einen l�cherlichen Versuch, die \fq{Cracker} zu �rgern, der aber seinen Zweck
verfehlt hat und nur den Benutzer �rgert.
